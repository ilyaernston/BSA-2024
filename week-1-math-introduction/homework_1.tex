\documentclass{article}
\usepackage{amsfonts}
\usepackage{graphicx} % Required for inserting images

\title{BSA: Homework Assignment 1}
\author{Ilia Ernston}
\date{November 2024}

\begin{document}

\maketitle

\section{Calculus}
\subsection{Question 1}
Let $ f(x) = x^3 - 6x^2 + 9x - 4 $. 
\begin{enumerate}
    \item Find $ f'(x) $, the first derivative of $ f(x) $.
    \item Determine the critical points of $ f(x) $ by solving $ f'(x) = 0 $.
    \item Using the second derivative test, classify each critical point as a local maximum, minimum, or point of inflection.
\end{enumerate}
\subparagraph{Answer}
\begin{enumerate}
    \item $ f'(x) = 3x^2-12x+9 $
    \item $3x^2-12x+9=0 \rightarrow 3(x^2-4x+3)=0 \rightarrow \left[ \begin{array}{cc} x=1 \\ x = 3 \end{array} \right]$
    \item $f''(x)=6x-12$

$f''(x_1)=6(1)-12=-6<0 \rightarrow x_1 $ is local maximum

$f''(x_2)=6(3)-12=6>0 \rightarrow x_2 $ is local maximum
\end{enumerate}

\subsection{Question 2}
Evaluate the following indefinite integral:
$$\int \frac{2x^3 - 5x + 3}{x^2} dx $$
\subparagraph{Answer}
$$\int \frac{2x^3 - 5x + 3}{x^2} dx = $$
$$=\int \frac{2x^3}{x^2} dx -\int \frac{5x}{x^2} dx + \int \frac{3}{x^2} dx = \int 2x dx -\int \frac{5}{x} dx + \int \frac{3}{x^2}dx = $$
$$ = x^2-5\ln \left|x\right|-\frac{3}{x}+C $$

\subsection{Question 3}
Compute the definite integral:
$$ \int_0^3 (3x^2 - 4x + 2) dx $$
Then interpret the result as the area under the curve of $ f(x) = 3x^2 - 4x + 2 $ over the interval $ [0, 3] $
\subparagraph{Answer}
$$ \int_0^3 (3x^2 - 4x + 2) dx = \left[ 3 \frac{x^3}{3}-4 \frac{x^2}{2}+2x \right] ^3_0 = (27-18+6)-(0) = 15$$ 
i.e. the area under the curve of $ f(x) = 3x^2 - 4x + 2 $ over the interval $ [0, 3] $ equals 15.

\subsection{Question 4}
Let $ g(x) = x e^x $.
\begin{enumerate}
    \item Find $ g'(x) $ using the product rule
    \item Determine the integral $ \int x e^x dx $ by using integration by part
\end{enumerate}
\subparagraph{Answer}
\begin{enumerate}
    \item $$g'(x) = \frac{d}{dx}(x e^x)$$
    applying the product rule:
    $$\frac{dx}{dx}(e^x) + \frac{d}{dx}(e^x)x = 1e^x+e^x x  = e^x + e^xx $$
    \item applying integration by parts to $ x e^xdx $:
    $$\int x e^x dx = xe^x-\int e^x dx = xe^x- e^x + C$$
\end{enumerate}

\subsection{Question 5}
Suppose $ h(x) = sin(x) e^{3x} $.
\begin{enumerate}
    \item Calculate $ h'(x) $.
    \item Compute $ \int sin(x) e^{3x} dx $ using integration by parts twice.
\end{enumerate}
\subparagraph{Answer}
\begin{enumerate}
    \item  $$h'(x) = \frac{d}{dx} sin(x)e^{3x}$$
    applying the product rule:
    $$\frac{d}{dx}(\sin (x))e^{3x}+\frac{d}{dx}(e^{3x})\sin (x) = \cos (x)e^{3x}+e^{3x} 3\sin (x)$$
    \item applying integration by parts to $ sin(x) e^{3x}dx$:
    $$\int sin(x) e^{3x} dx = -e^{3x}\cos (x)-\int \:-3e^{3x}\cos (x)dx = -e^{3x}\cos (x)-\left(-3 \int e^{3x}\cos (x)dx\right)$$
    applying integration by pars to $ e^{3x}\cos (x)dx$:
    $$ \int e^{3x}\cos (x)dx = e^{3x}\sin (x)-\int \:e^{3x} 3\sin (x)dx = e^{3x}\sin (x)-3 \int e^{3x}\sin (x)dx$$
    thus:
    $$\int e^{3x}\sin (x)dx=-e^{3x}\cos (x)+3\left(e^{3x}\sin (x)-3 \int \:e^{3x}\sin (x)dx\right)$$
    isolate $e^{3x}\sin \left(x\right)dx$:
    $$\frac{3e^{3x}\sin \left(x\right)}{10}-\frac{e^{3x}\cos \left(x\right)}{10}= \frac{e^{3x} (3\sin (x) - \cos (x))}{10}+C$$
\end{enumerate}

\section{Linear Algebra}
\subsection{Question 6}
Consider the following system of linear equations:
$$2x + 3y - z = 4$$
$$x - 2y + 4z = -3$$
$$3x + y + 2z = 5$$
\begin{enumerate}
    \item Write the system in matrix form $ AX = B $.
    \item Find the determinant of $ A $.
    \item If the determinant is non-zero, use the inverse of $ A $ to solve for $ X $.
\end{enumerate}
\subparagraph{Answer}
\begin{enumerate}
    \item $$ \left[ \begin{array}{ccc} 2 & 3 & -1 \\ 1 & -2 & 4 \\ 3 & 1 & 2 \end{array} \right] \left[ \begin{array}{c} x \\ y \\ z \end{array} \right] = \left[ \begin{array}{c} 4 \\ -3 \\ 5 \end{array} \right]$$
    where:
    $$\mathbf{A} = \left[ \begin{array}{ccc} 2 & 3 & -1 \\ 1 & -2 & 4 \\ 3 & 1 & 2 \end{array} \right], 
    \mathbf{B}= \left[ \begin{array}{c} 4 \\ -3 \\ 5 \end{array} \right], 
    X=\left[ \begin{array}{c} x \\ y \\ z \end{array} \right] $$
    \item $$\det (\mathbf{A})= 2 \cdot \begin{array}{|cc|} -2 & 4 \\ 1 & 2 \end{array} - 3 \cdot \begin{array}{|cc|} 1 & 4 \\ 3 & 2 \end{array} -1 \cdot \begin{array}{|cc|} 1 & -2 \\ 3 & 1 \end{array} = $$
    $$= 2(-2*2) - 2(4*1) - 3(1*2) + 3(4*3) - (1*1) + (-2*3)=7$$ 
    \item  $\det (\mathbf{A})= 7 \neq 0 \rightarrow \mathbf{A}$ is invertible and $X = \mathbf{A}^{-1}\mathbf{B}$.
    $$\mathbf{A}^{-1}= \left[\begin{array}{ccc|ccc}
      2 & 3 & -1 & 1 & 0 & 0 \\ 1 & -2 & 4 & 0 & 1 & 0\\ 3 & 1 & 2 & 0 & 0 & 1 
      \end{array}\right] \sim \left[\begin{array}{ccc|ccc}
      1.14 & -1 & 1.43 & 1 & 0 & 0 \\ 1.43 & 1 & 1.29 &  0 & 1 & 0\\ 1 & 1 & -1 & 0 & 0 & 1 
      \end{array}\right] $$
      $$
      X = \left[\begin{array}{ccc}
      1.14 & -1 & 1.43 \\ 1.43 & 1 & 1.29 \\ 1 & 1 & -1 
      \end{array}\right] 
      \left[ \begin{array}{c} 4 \\ -3 \\ 5 \end{array} \right]= \left[ \begin{array}{c} 5.57 \\ -3.71 \\ -4 \end{array} \right] $$
\end{enumerate}

\subsection{Question 7}
Given the matrices:
$$
\mathbf{A} = \left[ \begin{array}{cc} 2 & -1 \\ 0 & 3 \end{array} \right], \mathbf{B}= \left[ \begin{array}{cc} 4 & 1 \\ -2 & 5 \end{array} \right]
$$
\begin{enumerate}
    \item Compute $ \mathbf{A} + \mathbf{B} $ and $ \mathbf{A} - \mathbf{B} $
    \item Find the product $ \mathbf{A}\mathbf{B} $ and $ \mathbf{B}\mathbf{A} $
    \item Is $ \mathbf{A}\mathbf{B} = \mathbf{B}\mathbf{A} $? Justify your answer.
\end{enumerate}
\paragraph{Answer}
\begin{enumerate}
    \item $$\mathbf{A} + \mathbf{B} = \left[ \begin{array}{cc} 2+4 & -1+1 \\ 0-2 & 3+5 \end{array} \right] = \left[ \begin{array}{cc} 6 & 0 \\ -2 & 8 \end{array} \right]$$
    $$\mathbf{A} - \mathbf{B} = \left[ \begin{array}{cc} 2-4 & -1-1 \\ 0+2 & 3-5 \end{array} \right] = \left[ \begin{array}{cc} -2 & -2 \\ 2 & -2 \end{array} \right]$$
    \item $$\mathbf{A}\mathbf{B} = \left[ \begin{array}{cc} 2*4-1(-2) & 2*1-1*5 \\ 0*4+3(-2) & 0*1+3*5 \end{array} \right] = \left[ \begin{array}{cc} 10 & -3 \\ -6 & 15 \end{array} \right]
    $$
    $$\mathbf{B}\mathbf{A} = \left[ \begin{array}{cc} 8 & -1 \\ -4 & 17 \end{array} \right]
    $$
    \item Avobe results prove that $ \mathbf{A}\mathbf{B} \neq \mathbf{B}\mathbf{A} $. This might be expected since matrix multiplication is non commutative in general case.
\end{enumerate}

\subsection{Question 8}
Let $ \mathbf{C} $ be a $ 3 \times 3 $ matrix:
$$\mathbf{C} = \left[ \begin{array}{ccc} 1 & 2 & 3 \\ 0 & -1 & 4 \\ 2 & 1 & 0 \end{array} \right]$$
\begin{enumerate}
    \item Find the transpose of $ \mathbf{C} $, denoted $ \mathbf{C}^T $
    \item Calculate $ \mathbf{C} + \mathbf{C}^T $
    \item Determine if $ \mathbf{C} + \mathbf{C}^T $ is symmetric.
\end{enumerate}
\paragraph{Answer}
\begin{enumerate}
    \item $$\mathbf{C}^T = \left[ \begin{array}{ccc} 1 & 0 & 2 \\ 2 & -1 & 1 \\ 3 & 4 & 0 \end{array} \right] $$
    \item $$\mathbf{C} + \mathbf{C}^T = \left[ \begin{array}{ccc} 1+1 & 2+0 & 3+2 \\ 0+2 & -1-1 & 4+1 \\ 2+3 & 1+4 & 0+0 \end{array} \right] = \left[ \begin{array}{ccc} 2 & 2 & 5 \\ 2 & -2 & 5 \\ 5 & 5 & 0 \end{array} \right]$$
    \item $ \mathbf{C} + \mathbf{C}^T = (\mathbf{C} + \mathbf{C}^T)^T \rightarrow \mathbf{C} + \mathbf{C}^T$ is symmetric
\end{enumerate}

\subsection{Question 9}
A matrix $ \mathbf{D} $ is defined as:
$$\mathbf{D} = \left[ \begin{array}{ccc} a & 0 & 0 \\ 0 & b & 0 \\ 0 & 0 & c \end{array} \right]$$
\begin{enumerate}
    \item Describe the type of matrix $ \mathbf{D} $ and explain its properties.
    \item If $ \mathbf{D} $ is a diagonal matrix, find its determinant.
    \item Explain the significance of a diagonal matrix in terms of linear transformations.
\end{enumerate}
\paragraph{Answer}
\begin{enumerate}
    \item $ \mathbf{D} $ is a diagonal matrix, as all its non-diagonal elements are $0$.
    
    Main properties of diagonal matrices:
    \begin{itemize}
        \item Adding or multiplying diagonal matrices results in another diagonal matrix
        \item A diagonal matrix is invertible if all diagonal entries are nonzero such that $ \mathbf{D}^{-1} $ is also diagonal with entries $\frac{1}{a}, \frac{1}{b}, \frac{1}{c}$.
        \item The diagonal entries are also the eigenvalues of a diagonal matrix.
        \item The eigenvectors of diagonal matrix correspond to the standard basis vectors in the respective dimensions.
        \item For any $n \in \mathbb{Z}$ $ \mathbf{D}^n$ is diagonal, with diagonal entries $a^n, b^n, c^n$.
    \end{itemize}
    \item The determinant of a diagonal matrix is the product of its diagonal entries:
    $$ \det (\mathbf{D}) = a*b*c$$
    \item \begin{itemize}
        \item A diagonal matrix represents scaling along the coordinate axes.
        \item If $\mathbf{A}$ is diagonalazible, it can be expressed as an eigendecomposition $\mathbf{P}\mathbf{D}\mathbf{P}^{-1}$ where $\mathbf{D}$ is diagonal matrix with the eigenvalues of $\mathbf{A}$ as diagonal entries.
    \end{itemize}
\end{enumerate}

\subsection{Question 10}
Let $ \mathbf{E} $ be a $ 2 \times 2 $ matrix defined as:
$$
\mathbf{E} =  \left[ \begin{array}{cc} 3 & 4 \\ 2 & 1 \end{array} \right]
$$
\begin{enumerate}
    \item Find the eigenvalues of $\mathbf{E}$
    \item For each eigenvalue, find the corresponding eigenvector.
    \item Explain how the eigenvalues and eigenvectors can be used to understand transformations represented by $\mathbf{E}$
\end{enumerate}
\paragraph{Answer}
\begin{enumerate}
    \item The eigenvalues $\lambda$ can be found by solving the characteristic polynomial $ \det (\mathbf{E} - \lambda \mathbf{I})=0 $:
    $$ \begin{array}{|cc|} 3-\lambda & 4 \\ 2 & 1-\lambda \end{array} = 0$$
    $$ (3-\lambda)(1-\lambda)-(2)(4)=0 $$
    $$ \lambda^2 - 4\lambda - 5 =0$$
    $$ \begin{array}{c} \lambda_1 = 5 \\ \lambda_2 = -1 \end{array} $$
    \item For given eigenvalue $\lambda$ eigenvector $\mathbf{v}$ is found by substitution $\lambda$ into $(\mathbf{E} - \lambda \mathbf{I})\mathbf{v}=0$. 
    
    Thus for $\lambda_1=5$:
    $$ (\mathbf{E} - 5 \mathbf{I})\mathbf{v}=0$$
    $$ \left[ \begin{array}{cc} 3-5 & 4 \\ 2 & 1-5\end{array} \right] \left[ \begin{array}{c} x \\ y \end{array} \right] = \left[ \begin{array}{c} 0 \\ 0 \end{array} \right]$$
    $$ \left[ \begin{array}{cc} -2 & 4 \\ 2 & -4\end{array} \right] \left[ \begin{array}{c} x \\ y \end{array} \right] = \left[ \begin{array}{c} 0 \\ 0 \end{array} \right] \rightarrow -2x+4y=0 \rightarrow x=2y$$
    then $\mathbf{v}_1 = \left[ \begin{array}{c} 2 \\ 1 \end{array} \right]$;
    
    for $\lambda_2=-1$:
    $$ (\mathbf{E} +1 \mathbf{I})\mathbf{v}=0$$
    $$ \left[ \begin{array}{cc} 3+1 & 4 \\ 2 & 1+1\end{array} \right] \left[ \begin{array}{c} x \\ y \end{array} \right] = \left[ \begin{array}{c} 0 \\ 0 \end{array} \right]$$
    $$ \left[ \begin{array}{cc} 4 & 4 \\ 2 & 2\end{array} \right] \left[ \begin{array}{c} x \\ y \end{array} \right] = \left[ \begin{array}{c} 0 \\ 0 \end{array} \right] \rightarrow 4x+4y=0 \rightarrow x=-y$$
    then $\mathbf{v}_2 = \left[ \begin{array}{c} -1 \\ 1 \end{array} \right]$.
    \item \begin{itemize}
        \item The eigenvalues represent scaling factors of the transformation $\mathbf{E}$.
        \item The eigenvectors represent the directions that remain unchanged (except for scaling) under the transformation $\mathbf{E}$.
    \end{itemize}
\end{enumerate}

\section{Probability Theory}
\subsection{Question 11}
In a population, $30\%$ of individuals have a specific genetic marker. A test for this marker has a $95\%$ accuracy rate for individuals with the marker and an $80\% $ accuracy rate for those without the marker.,
\begin{enumerate}
    \item Define the events and write out the given probabilities.
    \item Using Bayes' theorem, calculate the probability that an individual actually has the marker given a positive test result.
\end{enumerate}
\paragraph{Answer}
Let 

$M$ define that individual has the genetic marker;

$\neg M$ - individual does not have the genetic marker;

$T^+$ - test result is positive;

$T^-$ - test result is negative.

Then:

$$P(M) = 0.3 \rightarrow P(\neg M) = 1-0.3 = 0.7$$

$$P(T^+|M) = 0.95 \rightarrow P(T^-|M) = 1 - 0.95 = 0.05$$

$$P(T^-|\neg M)=0.8 \rightarrow P(T^+|\neg M) = 1-0.8=0.2 $$
according to Bayes' theorem:
$$P(M|T^+)=\frac{P(T^+|M)P(M)}{P(T^+)}$$
then total probability of a positive test result:
$$P(T^+)=P(T^+|M)P(M)+P(T^+|\neg M)P(\neg M)=$$
$$=(0.95)(0.3)+(0.2)(0.7)=0.425$$
substituting to initial formula:
$$P(M|T^+)=\frac{(0.95)(0.3)}{0.425}\approx 0.6706 = 67.06\%$$

\subsection{Question 12}
A die is rolled 12 times. Let $ X $ be the random variable representing the number of times a "5" appears.
\begin{enumerate}
    \item What type of probability distribution does $ X $ follow?
    \item Calculate the probability that "5" appears exactly 4 times.
    \item Determine the expected number of times "5" will appear in 12 rolls.
\end{enumerate}
\paragraph{Answer}
\begin{enumerate}
    \item \begin{enumerate}
        \item Number of trials is fixed;
        \item Each trial is independent;
        \item There are only two outcomes: success (i.e. rolling a "5") or failure;
        \item The probability of success is constant for each trial.
    \end{enumerate}
    Thus $X \sim$ Binomial.
    \item If "5" appears exactly 4 times, than:
    $n=12,p=\frac{1}{6},k=4$.
    
    Since probability is binomial, probability that "5" appears exactly 4 times is defined as:
    $$P(X=k)= \left( \begin{array}{c} n \\ k \end{array} \right)p^k(1-p)^{n-k} $$
    $$P(X =4)=\left( \begin{array}{c} 12 \\ 4 \end{array} \right) \left(\frac{1}{6}\right)^4 \left(\frac{5}{6}\right)^8= \frac{12!}{4!(12-4)!}\left(\frac{1}{1296}\right) \left(\frac{390625}{1679616}\right) \approx$$
    $$ \approx 0.0888 = 8.88\%$$
    \item The expected value $\mathbb E[X]$ for a binomial random variable is given by:
    $$\mathbb E[X]=np$$
    Substituting parameters from given distribution:
    $$\mathbb E[X]=12\frac{1}{6}=2$$
\end{enumerate}

\subsection{Question 13}
A bookstore averages 3 customer visits per hour. Assume the number of customer visits per hour follows a Poisson distribution.
\begin{enumerate}
    \item Calculate the probability that exactly 5 customers visit in a given hour.
    \item Find the probability that at least 1 customer visits in a given hour.
\end{enumerate}
\paragraph{Answer}

Given number of customers follows Poisson distribution, the probability mass function is given by:
$$P(X=k)=\frac{\lambda^k e^{-\lambda}}{k!}$$
where $\lambda$ is mean number of customers per hour, $k$ is the number of customer visits.
\begin{enumerate}
    \item For $\lambda = 3$ and $k = 5$:
    $$P(X=5)=\frac{3^5 e^{-3}}{5!} \approx \frac{243 * 0.0498}{120} \approx 0.1009 = 10.09\%$$
    \item Probability of at least 1 customer is given by $P(X\geq 1)$. Since $k \in \mathbb{N}$, complement of $P(X\geq 1)$ is $P(X= 0)$.
    $$P(X=0)=\frac{3^0 e^{-3}}{0!} =\frac{1 * e^{-3}}{1} = e^{-3} \approx 0.0498$$
    $$P(X\geq 1) = 1- P(X=0)= 0.9502 = 95.02\% $$
\end{enumerate}

\subsection{Question 14}
Assume a continuous random variable $ Y $ follows a normal distribution with mean $ \mu = 50 $ and standard deviation $ \sigma = 5 $.
\begin{enumerate}
    \item Calculate the probability that $ Y $ takes a value between 45 and 55.
    \item Find the value $ y $ such that $ P(Y \leq y) = 0.85 $.
\end{enumerate}

\paragraph{Answer}
\begin{enumerate}
    \item Probability can be evaluated for the standardized value $Z$, given by:
    $$Z = \frac{Y - \mu}{\sigma}$$
    For $Y=45$ (given $Y$ is normally distributed):
    $$Z_45 = \frac{45 - 50}{5}=-1$$
    For $Y=55$:
    $$Z_55 = \frac{55 - 50}{5}=1$$
    From standard normal tables:
    $$P(Z \leq 1) \approx 0.8413, P(Z \leq-1) \approx 0.1587 $$
    $$P(Z_45 \leq Y \leq Z_55) = P(Z \leq 1) - P(Z \leq-1) = 0.8413 - 0.1587 = $$
    $$= 0.6826 = 68.26\%$$
    \item We nned to evaluate $y$ for
    $$P(Y \leq y)=0.85$$
    From the standard normal distribution table for $Y=0.85$ $Z \approx 1.036$.
    
    Since $Y=\mu+Z* \sigma$:
    $$y=50+(1.036)(5)=50+5.18=55.18$$
\end{enumerate}

\subsection{Question 15}
A company's product quality control process involves two independent tests. Each test has a $98\%$ probability of detecting a defect if it is present.
\begin{enumerate}
    \item What is the probability that both tests detect a defect when it is present?
    \item What is the probability that at least one test detects a defect when it is present?
    \item If the company requires both tests to detect a defect before rejecting a product, calculate the probability that a defective product will be rejected.
\end{enumerate}

\paragraph{Answer}
\begin{enumerate}
    \item Let:
    
    $P(A)=0.98$ be the probability that the first test detects a defect;
    
    $P(B)=0.98$ be the probability that the second test detects a defect.
    
    Since the tests are independent, the probability that both tests detect a defect is given by:
    $$P(A \cap B) = P(A) * P(B) = 0.98*0.98=0.9604=96.04\%$$
    \item The complement of "at least one test detects a defect" is "neither test detects a defect". 
    The probability that a test fails to detect a defect is:
    $$P(\neg A)=P(\neg B)=1-P(A)=1-P(B)=1-0.98=0.02$$
    The probability that neither test detects a defect is:
    $$P(\neg A \cap \neg B) = P(\neg A) * P(\neg B) = 0.02*0.02=0.0004$$
    Thus, the probability that at least one test detects a defect is:
    $$P(\neg A \cup \neg B)=1-P(\neg A \cup \neg B)=1-0.0004 = 0.9996 = 99.96\%$$
    \item The company requires both tests to detect a defect before rejecting a product. This is the same as the probability that both tests detect a defect:
    $$P(A \cap B)=0.9604=96.04\%$$
\end{enumerate}
\end{document}